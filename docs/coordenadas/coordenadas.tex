\documentclass[a4paper]{article}
\usepackage[spanish]{babel}
\usepackage[latin1]{inputenc}
\parindent 0mm
\parskip 3mm
\begin{document}

\section*{Cambio de coordenadas GPS}

Para realizar un cambio de coordenadas de las actuales $(x_1,y_1)$ a  las que es capaz de entender Google Maps $(x_1',y_1')$, habr� que resolver el sistema de ecuaciones lineales \ref{eq:cambio_coordenadas}


\begin{equation}
\left.
\begin{array}{rcl}
x_1'&=&a x_1+b\\
y_1'&=&c y_1+d
\end{array}
\right\}
\label{eq:cambio_coordenadas}
\end{equation}

Para obtener los valores de $a, b ,c$ y $d$, dado que es un sistema sobredimensionado, ser� necesario otro par de coordenadas:

\begin{equation}
\left.
\begin{array}{rcl}
x_2'&=&a x_2+b\\
y_2'&=&c y_2+d
\end{array}
\right\}
\label{eq:segundo_par}
\end{equation}


Resolviendo:

\begin{eqnarray}
a&=&\frac{x_1'-x_2'}{x_1-x_2}\\
b&=&x_1'-a x_1\\
c&=&\frac{y_1'-y_2'}{y_1-y_2}\\
d&=&y_1'-c y_1
\end{eqnarray}

�Qu� problema me he encontrado? Pues que las coordenadas $(x_1,y_1)$ no se corresponden con la direcci�n postal registrada y, por tanto, tampoco con las coordenadas GPS $(x_1',y_1')$ que se obtienen de Google.

�Por d�nde pienso continuar? Pues por coordenadas que pueda identificar en la nube de puntos, como el cruce de la Marga, el principio de Marqu�s de la Hermida o la pen�nsula de Magdalena y, en esos casos s�, aplicar las ecuaciones planteadas anteriormente para poder superponer un plano de la ciudad.

Si alguien tiene una idea mejor, por favor, que me la comente.

\end{document}